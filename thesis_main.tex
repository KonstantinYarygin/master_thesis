\documentclass[master,14pt,subf,href,colorlinks=true
%,times        % шрифт Times как основной
%,fixint=false % отключить прямые знаки интегралов
]{disser}

\usepackage[utf8]{inputenc}
\usepackage{graphicx}
\usepackage{mathtools}

\usepackage[a4paper, mag=1000, includefoot, left=3cm, right=2cm, top=2cm, bottom=2cm, headsep=1cm, footskip=1cm]{geometry}
\usepackage[T2A]{fontenc}
\usepackage[english,russian]{babel}
\ifpdf\usepackage{epstopdf}\fi

% Номера страниц сверху и по центру
%\def\headfont{\small}
%\pagestyle{headcenter}
%\chapterpagestyle{empty}

% Точка с запятой в качестве разделителя между номерами цитирований
%\setcitestyle{semicolon}

% Использовать полужирное начертание для векторов
\let\vec=\mathbf

% Включать подсекции в оглавление
\setcounter{tocdepth}{2}

\graphicspath{{figure/}}


\begin{document}
\input{d_thesis_title}
\tableofcontents % это оглавление, которое генерируется автоматически
% \input{review} % это титульный лист
\end{document}
% рекомендуется оформлять текст шрифтом Times New Roman от 12 до 14 pt
% межстрочный интервал 1.5
% выравнивание в абзацах по ширине
% поля на странице: левое – 30 мм, остальные 20 мм.
